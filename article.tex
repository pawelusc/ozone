\documentclass[12pt,pra,aps,superscriptaddress]{revtex4-2}
\usepackage{amsmath,amssymb,physics}
\usepackage{hyperref} % Clickable links in the document
\usepackage{graphicx} % For figures with images
\usepackage{natbib} % bibliography
\setcitestyle{super, sort&compress, comma} % chemistry style
\usepackage{hyperref} % Clickable links in the document and their colors
\hypersetup{
    colorlinks=true, % false: boxed links; true: colored links
    linkcolor=cyan, % color of internal links
    citecolor=magenta, % color of links to bibliography
    filecolor=magenta, % color of file links
    urlcolor=cyan, % color of external links
    runcolor=cyan
}

\begin{document}

\title{Vibronic Coupling Effects in the Photoelectron Spectrum of Ozone: A
Coupled-Cluster Approach}

\author{Pawe{\l} W{\'o}jcik}
\affiliation{Department of Chemistry, University of Southern California, Los Angeles, California 90089, USA}
\affiliation{Laboratory of Theoretical Chemistry, Institute of Chemistry, ELTE
E{\"o}tv{\"o}s Lor{\'a}nd University, P{\'a}zm{\'a}ny P{\'e}ter stny. 1/A,
Budapest, Hungary}

\author{Anna I. Krylov}
\affiliation{Department of Chemistry, University of Southern California, Los Angeles, California 90089, USA}

\author{Hannah Reisler}
\affiliation{Department of Chemistry, University of Southern California, Los Angeles, California 90089, USA}

\author{P{\'e}ter G. Szalay}
\affiliation{Laboratory of Theoretical Chemistry, Institute of Chemistry, ELTE
E{\"o}tv{\"o}s Lor{\'a}nd University, P{\'a}zm{\'a}ny P{\'e}ter stny. 1/A,
Budapest, Hungary}

\author{John F. Stanton}
\email[Electronic address: ]{jfstanton@gmail.com}
\affiliation{Quantum Theory Project, Departments of Chemistry and Physics, University of Florida, Gainesville, FL, USA 32611}
\affiliation{Laboratory of Theoretical Chemistry, Institute of Chemistry, ELTE
E{\"o}tv{\"o}s Lor{\'a}nd University, P{\'a}zm{\'a}ny P{\'e}ter stny. 1/A,
Budapest, Hungary}

\date{\today}

\begin{abstract}
One of the most important areas of application of equation-of-motion coupled
cluster (EOM-CC) theory is to the prediction, simulation and analysis of
various types of electronic spectra.   In this work, the EOM-CC method known
as EOMIP-CC is applied to the closely lying and coupled pair of states of the
ozone cation -- ${\tilde X}^2A_1$ and ${\tilde A}^2B_2$ -- using sophisticated
approaches that extend to the full singles, doubles, triples and quadruples
model (EOMIP-CCSDTQ).   Combined with a venerable and powerful method for
calculating vibronic spectra from the Hamiltonian produced by EOMIP-CC
calculations, this research provides a spectrum that is in good agreement with
the photoelectron spectrum of ozone.   An important result here is that the
calculations suggest that the adiabatic gap separating these two electronic
states is somewhat smaller than currently thought; an assignment of the
simulated spectrum together with the more precise band positions of the
experimental measurements suggests that this energy gap is 1366$\pm$65
cm$^{-1}$.
\end{abstract}

\maketitle

\section{Introduction}

Amongst the brotherhood of triatomic molecules, it cannot be argued that water
(H$_2$O) is the most important, the most highly studied, and the most well
understood.  Beyond H$_2$O, there are many triatomic molecules that have an
environmental, technological or biological importance, have been subjected to
many studied and are understood to various levels of detail.  Perhaps the most
interesting such case is ozone (O$_3$), which has a vast number of important
properties, a very rich history of study \cite{chappuis}, and – unlike the
relatively simple water molecule – a profound quantum-mechanical complexity
\cite{Babikov:anomalousOzone:2003}.  In the latter context, while we think of
(and an NMR experiment would reveal) as two distinct kinds of oxygen Atoms in
ozone, the full molecular Hamiltonian does not distinguish between the two
types, with the three equivalent structures separated by a barrier that lies
tantalizingly close to the O$_3$ $\rightarrow$ O$_2$ + O dissociation
threshold (102.4 kJ/mol) \cite{Ruscis:ATcT:2022}. In reality, the energy
levels of ozone all have a near triple ($e+a$) degeneracy, albeit with a
tunneling splitting so small that it can be ignored, along with a
semi-infinite lifetime (despite opposing opinion
\cite{Boggs:BerryOzone:2006}).   More than a half century ago, this intriguing
aspect of the ozone molecule was first discussed by Berry
\cite{Berry:Ozone:1960}.

Beyond the structural aspects of ozone, other mysteries center around this
curious molecule.  For example, the distribution of the eighteen distinct
$^{16}$O/$^{17}$O/$^{18}$O isotopologues in the Earth’s atmosphere differs
from what is expected based on the natural isotopic abundance, a puzzle that
has been open for more than three decades
\cite{Mauersberger:OzoneMystery:1990}. 

Amongst quantum chemists, ozone has a notorious history, with its strong
biradical character causing significant difficulties in calculations of its
ostensibly simple ground state molecular properties.  An early 1989 study
\cite{Stanton:Ozone:1989, Stanton:Ozone:1989b} by the Bartlett group and
collaborators found that the CCSD+T(CCSD) method predicted that the molecular
equilibrium structure of ozone would have $C_s$ symmetry (that is, the
asymmetric stretching harmonic frequency predicted by this method was
imaginary), a finding that led to a search for better treatment of
non-iterative triple excitations, ultimately leading to the well-known CCSD(T)
treatment \cite{Raghavachari:89, Urban:ccsd(t):1985, GenrefCCSD(T):93}. While
CCSD(T) and higher-level coupled-cluster methods available today
\cite{kucharski:ccsdtq:1992, Kallay:CCHigh, Matthews:ncc:2015} do a good job
in predicting the molecular vibrational potential, an elaborate multireference
configuration interaction study by Dawes {\it et al.} has done an excellent
job on the entire ground-state surface out to the dissociation limit
\cite{Dawes:ozone:2013}. As such, the quantum chemical understanding and
fidelity for the ground electronic state of O$_3$ is now at a mature level.

Qualitatively, the challenge posed to electronic structure by ozone ultimately
arises from its closely spaced highest-occupied ($b_2$) and lowest-unoccupied
($a_1$) molecular orbitals (HOMO and LUMO, respectively).  The two electron
configurations $[\cdots]b_2^2 a_1^0$ and $[\cdots]b_2^0 a_1^2$ mix strongly,
posing the aforementioned challenges with (especially single-reference)
quantum chemical methods.   Another consequence of the identity and energetic
proximity of these two orbitals is that the ozone cation (which is
isoelectronic to the NO$_2$ radical) has closely lying $^2A_1$ and $^2B_2$
electronic states.   Like the associated states in NO$_2$, both of these
states are plagued by orbital symmetry-breaking problems
\cite{Davidson:SymmBreak:76}, a problem that greatly complicates
quantum-chemical efforts to them.   One of the many accomplishments of the
Bartlett group has been the integral role played by them in the development of
equation-of-motion coupled-cluster theory \cite{Stanton:93:EOMCC,
Nooijen:EOMEA:95, Bartlett:CC_review:07} (EOM-CC, also known as linear
response coupled-cluster theory \cite{Koch:90:LinResp}). These methods provide
a very efficient and simple way to study certain classes of what are often
termed ``multireference problems" \cite{Krylov:EOMRev:07}, and are ideally
suited to studying many reactive intermediates, radicals, biradicals and
electronically excited states.   From a somewhat wider viewpoint, the
existence of closely spaced electronic states always carries the potential for
(possibly strong) vibronic coupling, a phenomenon that can play an important
role in molecular dynamics and spectroscopy.  Indeed, one of the great
successes of EOM-CC methods has been in their ability -- if and only if
combined with vibronic coupling models -- to enable high-quality simulations
of complicated electronic spectra.   Such work provides important insight into
the nature of vibronic coupling in molecular systems, as has been exemplified
by various application studies (for example, see Refs.~\cite{Stanton:NO3:07}
and~\cite{Koppel:02}).

Our contribution to this issue paying homage to the career and accomplishments
of R.J. Bartlett consists of an application of a vibronic coupling model
parametrized by EOM-CC calculations to the photoelectron spectrum of ozone.
We trust that this combination of methodology applied to a molecule that has
been extensively studied by the Bartlett group is an appropriate contribution
to this issue.

\begin{figure}
    \includegraphics[width = 16 cm]{./figures/ozone_intro}
    \caption{ 
        Two lowest states of the ozone cation and normal modes of ozone.
    }
    \label{fig:ozone_intro}
\end{figure}

\section{Methods}

Ozone is a $C_{2v}$ molecule (we are following the Mulliken's
convention~\cite{Mulliken:55:symnot} by placing the molecule in the $yz$ plane
and aligning the molecular symmetry axis with the $z$ direction) with three
normal modes: symmetric stretch, symmetric bend, and asymmetric stretch.  The
asymmetric stretch is of a $b_2$ symmetry. The two lowest electronic states of
the ozone cation are close in energy. The lower state is fully symmetric while
the higher one is of the $B_2$ symmetry, the same symmetry as the asymmetric
stretch. The close energetic separation and the matching symmetry of the
asymmetric stretch results in significant vibronic effects appearing in the
ozone photoelectron spectrum.  Figure~\ref{fig:ozone_intro} presents a
graphical summary of this information.

We simulate the vibronic states of the ozone cation using the model
Hamiltonian of K{\"o}ppel, Domcke, and Cederbaum (KDC
Hamiltonian).\cite{Cederbaum:LVC:84,KDC:81,Koppel:CIbookCh7:04} This is a
multi-state and multi-mode Hamiltonian defined in the basis of diabatic
states.  We consider the ozone cation in the basis of two quasi-diabatic
states coupled by one mode (mode number 3). The model also includes two
symmetric modes (modes number 1 and 2)
\begin{subequations}
    \begin{equation}
        H = H _0 \mathbf{1}
        +
        \begin{pmatrix}
            V ^{(1)}  & \lambda _3 Q _3\\
            \lambda _3 Q _3 & V ^{(2)}
        \end{pmatrix}
    \end{equation}
    \begin{equation}
        H _0 = 
        \frac{1}{2} \left(\sum _{i = 1}^3 
        - \omega _i \frac{\partial ^2}{\partial Q _i ^2 }\right)
        + \frac{1}{2}\omega _3 Q _3 ^2
    \end{equation}
    \begin{equation}
        V ^{(\alpha)} = 
        E ^{(\alpha)} 
        + \sum _{i,j,k,l \in \{1,2\}} 
            \kappa ^{(\alpha)} _i Q _i 
            + \kappa ^{(\alpha)} _{ij} Q _i Q _j 
            + \kappa ^{(\alpha)} _{ijk} Q _i Q _j Q _k 
            + \kappa ^{(\alpha)} _{ijkl} Q_i Q _j Q _k Q _l.
    \end{equation}
    \label{eq:KDC}
\end{subequations}
The Hamiltonian parameters are expanded around the equilibrium geometry of
ozone. $E ^{(\alpha)}$ are the vertical ionization energies calculated at that
geometry. $Q_i$ are the dimensionless normal coordinates of ozone. $\kappa$ are
the coefficients of expansion of the potential along the fully symmetric
coordinates. $\lambda$ are the linear diabatic couplings.  $\omega$ are the
harmonic frequencies of ozone.

We find the parameters that enter the KDC Hamiltonian using \emph{ab inito}
coupled-cluster (CC) methods and its equation-of-motion EOM-CC
extension.~\cite{Bartlett:CC_review:07, Krylov:EOMRev:07, Bartlett:Book:09,
Christiansen:EOMRev:11, Bartlet:EOMRev:12, Krylov:OSRev} We use the CC
truncated to singles and doubles (CCSD), singles, doubles and triples (CCSDT)
as well as singles, doubles, triples and quadruples
(CCSDTQ).\cite{Matthews:ncc:2015}  We use the EOM-CC for ionization potential
(EOM-IP).~\cite{StantonGauss:EOMIP:99} We use the Ichino, Gauss and Stanton
definition of quasi-diabatic states based on the EOM-CC method
(EOM-CC-QD)~\cite{Stanton:EOMIPdeg:09}. In all CC and EOM-CC calculations we
leave all electrons correlated, in other words, we do not use the
frozen-core approximation.

Using CCSDT/ANO1 we optimize the geometry of ozone and compute its harmonic
frequencies and normal coordinates.~\cite{Almlof:ANO,Almlof:ANO:1988} At the
same geometry we compute the linear diabatic coupling $\lambda$ using
EOM-IP-CCSD-QD/ANO1.  We use EOM-IP-CCSDT/ANO1 on a grid to find the
expansion coefficients $\kappa$.

The vertical ionization energies are calculated using a composite method. The
base value is the complete basis set (CBS) extrapolation of the
EOM-IP-CCSDT/cc-pCVnZ energies with n = 5, 6.~\cite{Woon:95:CCBS} These values
are augmented with two corrections: the $\Delta$Q correction in the cc-pwCVTZ
basis set and the relativistic correction calculated using
EOM-IP-CCSD/cc-pwCVTZ.~\cite{Dunning:02:p(w)CVnZ} We introduce an error
estimates to the reported vertical energies. For the error estimate of the
extrapolated CBS value we use half of the absolute value of the difference
between the best \emph{ab initio} value and the extrapolated value. For the
remaining corrections we use as the error estimate half of the absolute value
of the correction.

To compare the simulated photoelectron spectrum to the experimental one, we
use the oscillator strengths ratio A$_1$:B$_2$ of $1$:$1.35$.\cite{KDC:O3:92}
Additionally, the stick spectrum is broadened with the Lorenzian envelopes
normalized to the peaks' intensities
\begin{equation}
    f _{envelope}(x, x _i, I _i) = 
    \frac{I_i}{(x-x _i)^2 + (\gamma/ 2) ^2}.
    \label{eq:lorentzian}
\end{equation}
$x _i$ is the position of the spectral peak, $I _i$ is its intensity and
$\gamma$ is the peak's width.

We use the \textsc{xsim} program of Dr. Stanton to simulate the spectrum using
the basis of 50 harmonic states in each mode and 6000 iterations of the
Lanczos procedure. All remaining \emph{ab initio} calculations are completed
using \textsc{CFOUR}.~\cite{cfour, cfour:2020}


\section{Results}


The optimized geometry of ozone gives the bond lengths of $1.270$~\AA{} and
the bond angle of $116.9^\circ$. The two symmetric normal modes have
frequencies $1169$~cm$^{-1}$ and $724$~cm$^{-1}$, while the asymmetric stretch
has frequency equal to $1129$~cm$^{-1}$. The computed value of the linear
diabatic coupling constant $\lambda$ is $1394$~cm$^{-1}$. The vertical
ionization energy for the first excited state, $E^{(1)}$, computed using the
composite method described above is equal to $12.827$~eV. Our error estimate
for this value is $30$~meV, see Table~\ref{tab:vertical_ionization_energy} for
details.  We note that the convergence of the vertical ionization gap between
the two states is much faster and we estimate this value as equal to
$123\pm8$~meV, see Table~\ref{tab:vertical_gap} for details. 

\begin{table}
    \caption{
        Vertical ionization energies with the error estimates, eV.
    }
    \label{tab:vertical_ionization_energy}
    \begin{center}
        \begin{tabular}[c]{|l|rr|r|}
            \hline
            Contribution  & $^2$A$_1$ & $^2$B$_2$ & Error estimate \\ \hline
            CBS           & 12.872    & 12.981    & 0.02 \\
            $\Delta$Q     & -0.037    & -0.021    & 0.02 \\
            Relativistic  & -0.008    & -0.010    & 0.005 \\ \hline
            Sum           & 12.827    & 12.950    & 0.03 \\ \hline
        \end{tabular}
    \end{center}
\end{table}

\begin{table}
    \caption{
        The energy gap between vertical ionization energies of the $^2$A$_1$
        and $^2$B$_2$ states with error estimates, meV. See caption of
        Table~\ref{tab:vertical_ionization_energy}.
    }
    \label{tab:vertical_gap}
    \begin{center}
        \begin{tabular}[c]{|l|r|r|}
            \hline
            Contribution             &  Gap    & Error estimate \\ \hline
            EOM-CCSDT/CBS            &  108.8  & 0.9 \\
            $\Delta$Q/pwCVTZ         &   16.5  & 8 \\
            Relativistic/CCSD/pwCVTZ &   -2.2  & 1.1 \\ \hline
            Final value, meV         &  123    & 8 \\
            Final value, cm$^{-1}$   &  990    & 65 \\ \hline
        \end{tabular}
    \end{center}
\end{table}

\begin{figure}
\includegraphics[width = 8 cm]{figures/sim_vs_Dyke}
\caption{
    Comparison of the experimental (solid black line) and the simulated
    spectrum. In the simulated spectra the plot is generated with peak width
    $\gamma = 30$~meV. The simulated spectrum is shifted towards higher
    energies by $21$~meV.
}
\label{fig:sim_vs_dyke}
\end{figure}

Figure~\ref{fig:sim_vs_dyke} compares the simulated spectrum to the
experimental one taken from reference~\cite{dyke:O3:74}. A comparison of the
simulated and experimental spectra is presented on
Figure~\ref{fig:sim_vs_dyke} and shows a good match. Peak A is known to be a
hot band.~\cite{KDC:O3:92} The simulation reproduces well the consecutive
increase in the intensities of peaks B, C, D, and E. The spacing between these
peaks is also well reproduced. Drop in the intensity of peak F is also
captured by the simulation. Starting from peak G the simulation shows
discrepancy with the experiment. A sudden drop in the intensity past the peak
H is not observed in the simulation. We discuss a likely source of this
mismatch later.

\begin{figure}
    \includegraphics[width=10 cm]{figures/spectrum_overline}
    \caption{
        Simulated photoelectron spectrum of ozone. Bottom axis shows energy
        scale in eV. Top axis shows energy offset from the origin in
        cm$^{-1}$. Stick spectrum shows positions and intensities of all
        simulated states. Blue color corresponds to the oscillator strength
        originating from the $^2$A$_1$ basis state while the red ones
        correspond to $^2$B$_2$. Gray vertical lines with captions on top
        indicate positions of features as measured by the PFI-ZEKE
        experiment.~\cite{Willitsch:O3ZEKE:2005} 
        $D_0$ marks the dissociation threshold of O$_3^+$.
        Gray dotted line marks the energy of the minimum of the conical
        intersection (CI).
    }
    \label{fig:ozone_overlay}
\end{figure}

Our simulation allows for an additional element of analysis of the simulated
spectrum. Figure~\ref{fig:ozone_overlay} presents the same simulated spectrum
as in Figure~\ref{fig:sim_vs_dyke} with additional information on the
decomposition of the spectrum.  All lines that contribute to the spectrum are
marked individually and in two different colors. The spectrum shows that peaks
A and B are almost purely of the $^2$A$_1$ character. Starting from peak C
each the contributions from two states are equally important, it is also clear
that each peak of the low-resolution spectrum has in fact contributions from
many vibronic peaks. At the energy of about $2000$~cm$^{-1}$ above the origin
the density of vibronic peaks increases significantly. This value can be
compared to the minimum of the conical intersection, which our simulation
locates at $3174$~cm$^{-1}$ above the origin ($12.92$~eV). Our simulation is
incapable of accounting for dissociation of the molecule, therefore the we
expect a discrepancy with the experiment as the energy gets closer to the
dissociation threshold located at $4840$~cm$^{-1}$ above the
origin.~\cite{Willitsch:O3ZEKE:2005}

\begin{figure}
    \includegraphics[width=10cm]{figures/spectrum_assigned.pdf}
    \caption{
        Simulated photoelectron spectrum of ozone, but with no account given
        to the vibronic coupling effect.
    }
    \label{fig:no_coupling}
\end{figure}

We would like to assign the vibronic peaks from our simulation. To this end,
we run the simulation once again, this time, however we set the linear
diabatic constant to zero, i.e., $\lambda = 0$. With that change we can
reproduce the spectrum (at an equivalent level of theory) but in an artificial
case where there is no vibronic coupling. Figure~\ref{fig:no_coupling}
presents this spectrum. The non-coupled spectrum is easy to assign using the
labels that mark the symmetry of the electronic state, A$_1$ or B$_2$, and the
vibrational state ($\nu _1 \nu_2 \nu_3$), where $\nu _i$ is the number of
quanta in mode $i$ with $i=1$ for the symmetric stretch, $i=2$ for the
symmetric bend and $i=3$ for the asymmetric stretch. The assigned spectrum
shows progressions in the symmetric bend. There is one such progression in
each electronic states.  Additionally for each state, there is also another
progression in the symmetric bend with one excitation in the symmetric
stretch.

\begin{figure}
    \includegraphics[width=16 cm]{./figures/sim_vs_zeke}
    \caption{
        Comparison of the simulated spectrum with the PFI-ZEKE
        experiment.\cite{Willitsch:O3ZEKE:2005} The simulated spectrum was
        shifted by $21$~meV$ = 170$~cm$^{-1}$ towards higher energies.
    }
    \label{fig:sim_vs_zeke}
\end{figure}

We decompose the vibronic states of our main simulation in the basis of the
artificial, uncoupled states discussed in the previous paragraph.
Figure~\ref{fig:sim_vs_zeke} shows the assigned spectrum and
Table~\ref{tab:peak_assignment} lists the decomposition of all peaks in the
region of up to $3000$~cm$^{-1}$ with intensities larger than $10^{-3}$. We
compare the assigned spectrum against the high-resolution
pulsed-field-ionization zero-kinetic-energy (PFI-ZEKE) spectrum from
2005.\cite{Willitsch:O3ZEKE:2005} The PFI-ZEKE spectrum is the best source of
information on the location of the peaks, especially the origin, against which
align our simulation. The origin of the simulated spectrum is located
$170$~cm$^{-1}$ lower than the experimental origin which was experimentally
observed at $101,020.5$~cm$^{-1}$.\cite{Willitsch:O3ZEKE:2005} This is within
our error estimate for the vertical ionization energy
($30$~meV$=240$~cm$^{-1}$).  Once the origin locations are aligned the
comparison shows a good match between the simulation and the experiment.
Especially the states with an oscillator strength originating from the A$_1$
state are well aligned with the experimental features. 

States close to the origin of the band are similar to the non-coupled states.
The origin peak is of a clear A$_1(0,0,0)$ character, the first two
excitations in the symmetric bend, A$_1(0,1,0)$ and A$_1(0,2,0)$ are also very
similar to the non-coupled states, while the higher excitations in this
progression are showing large mixing. The same progression with one
vibrational quantum in the symmetric stretch is more interesting. The first of
its peaks is very well aligned with an experimentally visible feature, which
was previously assigned as the origin of the second state. The second peak in
this series, the first combination state A$_1(1,1,0)$, has very high
similarity to its uncoupled version. It is a good candidate to assign the
experimental, unassigned feature above $102,600$~cm$^{-1}$.

States with an oscillator strength from the B$_2$ state are on the other hand
not aligning well with the experiment. First such peak, close to the $12.64$~eV
mark on the Figure~\ref{fig:sim_vs_zeke}, is of a vibronic character and we
assign it as a A$_1$(0,0,1) state which gains intensity thanks to the coupling
to the B$_2$ state. This vibronic peak lies in the part of the spectrum marked
as pollution due to water. The origin of the B$_2$ state is very similar to
the uncoupled B$_2(0,0,0)$ state, but it is located in an empty area of the
experimental spectrum. The next peak, slightly above the $12.76$~eV mark on
Figure~\ref{fig:sim_vs_zeke}, corresponds to the one excitation of the symmetric
bend in the B$_2$ state, but as the label on the figure shows, it is only
about $55$\% similar to the uncoupled state.

\begin{figure}
    \includegraphics[width=16 cm]{figures/sim_vs_TarroniCarter}
    \caption{
        Comparison of the simulated spectrum to an earlier accurate simulation
        of Tarroni and Carter (assigned lines).~\cite{tarroni:O3:2011}
    }
    \label{fig:sim_vs_tarronicarter}
\end{figure}

Finally, we compare our simulation to an earlier accurate simulation, the work
of Tarroni and Carter from 2011.~\cite{tarroni:O3:2011} Both spectra show a
good match to the experiment and agree in the assignment of the first
progression in the bending mode. The previous simulation, however, shows
smaller spacing between other lines, showing much higher congestion of the
spectrum. Additionally the origin of the second state falls at lower energy
aligning well with the peak that was assigned in the PFI-ZEKE also as the
origin of the second state.  While an additional simulation of the line
intensities would allow for the most complete comparison to the experimental
spectrum, the strong alignment of the simulated peaks of a leading
A$_1$ character with the experimental features leads us to believe that our
simulation offers the best assignment of the spectrum to date.

\begin{table}
    \renewcommand{\arraystretch}{0.8}% Tighter
    \caption{Assignment and decomposition of the eigenvectors of the ozone
    cation.}
    \label{tab:peak_assignment}
    \begin{tabular}{|l|l|l|}
        \hline
        Peak position (cm$^{-1}$) & Assignment & Eigenvector \\
        \hline
$     0$ &  $A_1(000)$ & $ -0.99~A_1(000)$ \\
$   618$ &  $A_1(010)$ & $ +0.99~A_1(010)$ \\
$   915$ &  $A_1(001)$ & \\
$  1076$ &  $A_1(100)$ & $ -0.99~A_1(100)$ \\
$  1222$ &  $A_1(020)$ & $ -0.97~A_1(020)$ \\
$  1368$ &  $B_2(000)$ & $ +0.97~B_2(000)$ \\
$  1498$ &             & $ -0.10~B_2(000)$ \\
$  1684$ &  $A_1(110)$ & $ -0.96~A_1(110)$ \\
$  1796$ &  $A_1(030)$ & $ -0.87~A_1(030) -0.16~A_1(110) -0.13~A_1(020) +0.13~A_1(040)$ \\
$  1848$ &             & $ +0.31~A_1(030)$ \\
$  1921$ &  $B_2(010)$ & $ -0.74~B_2(010) +0.18~B_2(020) +0.11~B_2(000)$ \\
$  1977$ &             & $ +0.24~B_2(010)$ \\
$  2085$ &             & $ -0.52~B_2(010)$ \\
$  2132$ &             & $ +0.25~A_1(030) +0.24~A_1(040) +0.13~A_1(120) -0.13~A_1(050)$ \\
$  2275$ &  $A_1(120)$ & $ +0.86~A_1(120) +0.12~A_1(040)$ \\
$  2363$ &             & $ -0.62~A_1(040) +0.40~A_1(120) -0.17~A_1(030) +0.11~A_1(050)$ \\
$  2439$ &             & $ +0.47~A_1(040) +0.12~A_1(120)$ \\
$  2454$ &             & $ +0.53~B_2(020) +0.24~B_2(010) -0.17~B_2(030)$ \\
$  2576$ &             & $ +0.27~B_2(020)$ \\
$  2736$ &             & $ +0.60~B_2(020)$ \\
$  2886$ &  $A_1(130)$ & $ -0.81~A_1(130) +0.12~A_1(050) -0.10~A_1(120)$ \\
$  2974$ &             & $ -0.24~B_2(110) -0.16~B_2(020)$ \\
$  3047$ &  $A_1(050)$ & $ +0.76~A_1(050) +0.13~A_1(130)$ \\
        \hline
    \end{tabular}
\end{table}

\section{Summary and Conclusions} 

We have simulated the vibronic effects in the photoelectron spectrum of the
ozone.  We have used the KDC Hamiltonian\cite{Cederbaum:LVC:84, KDC:81,
Koppel:CIbookCh7:04} with the \emph{ab initio} parametrisation of the
quasi-diabatic states of Ichino, Gauss and Stanton.~\cite{Stanton:EOMIPdeg:09}
The results of our simulation match well the photoelectron spectrum from
1970s~\cite{dyke:O3:74} and the PFI-ZEKE spectrum from
2005.~\cite{Willitsch:O3ZEKE:2005} We present analysis of the spectra and
highlight the role of the vibronic coupling effects. Our simulation offers a
state-of-the-art insight into the spectrum and allows us to assign the
simulated peaks. Our assignment agrees well with the previously assigned
progression in the symmetric bend but we reassign some of the other features
of the spectrum. Simulated peaks that gain intensity from the B$_2$ state are
absent in the PFI-ZEKE spectrum which offers an interesting avenue for further
investigations. Modeling of the intensity of the PFI-ZEKE peaks is not
unlikely to bring insight into this problem.  Additionally our simulation does
not model dissociation of the ozone cation, which will play a role in the high
energy part of the spectrum.

\section{Acknowledgments} 

This project was initiated when three of the authors (P.W, P.G.S. and J.F.S)
were in Budapest, where J.F.S. was serving as a recipient of the John von
Neumann Award in STEM bestowed by the Fulbright Foundation. Additional
research presented here benefited from the NSF Center for Chemical Innovation
Phase I (grant no. CHE-2221453) and U.S. Department of Energy, Basic Energy
Sciences (grant no. DE-FG02-05ER15629).  All the authors of this research wish
Prof. Bartlett a happy ninth decade of life, and hope that he continues to
stimulate others in the field with his creative insights.

\clearpage

\bibliographystyle{prf}
\bibliography{allrefs}


\end{document}

% vim: tw=78 spell:
